% Created 2017-04-23 Sun 09:13
\documentclass[11pt]{article}
\usepackage[utf8]{inputenc}
\usepackage[T1]{fontenc}
\usepackage{fixltx2e}
\usepackage{graphicx}
\usepackage{grffile}
\usepackage{longtable}
\usepackage{wrapfig}
\usepackage{rotating}
\usepackage[normalem]{ulem}
\usepackage{amsmath}
\usepackage{textcomp}
\usepackage{amssymb}
\usepackage{capt-of}
\usepackage{hyperref}
\setcounter{secnumdepth}{3}
\author{weiwu}
\date{\today}
\title{}
\hypersetup{
 pdfauthor={weiwu},
 pdftitle={},
 pdfkeywords={},
 pdfsubject={},
 pdfcreator={Emacs 24.5.1 (Org mode 8.3.4)},
 pdflang={English}}
\begin{document}

\setcounter{tocdepth}{2}
\tableofcontents

t

\section{database = set of named relations(or tables)}
\label{sec:orgheadline2}
\begin{itemize}
\item set of columns
\item each tuple has a value for each column
\item each attribute has a type(or domain)
\end{itemize}
\subsection{schema}
\label{sec:orgheadline1}
\begin{itemize}
\item structural description of relations in dbname
\item instance
\item key: attribute whose value is unique in each tuple
\end{itemize}
\section{Querying Relational Database}
\label{sec:orgheadline4}
\subsection{Query Language}
\label{sec:orgheadline3}
\begin{itemize}
\item Relational Algebra
\end{itemize}
formal
\begin{itemize}
\item SQL
\end{itemize}
actual / implemented
\section{Relational Algebra}
\label{sec:orgheadline12}
\subsection{select, project, join}
\label{sec:orgheadline8}
\begin{itemize}
\item query on set of reltations produces relations as a result
\item use operators to filter, slice, combine results.
\item Select operator:
\end{itemize}
\(\sigma_condition(Expression)\)
\(\s GPA > 3.7 ^ HS < 1000 Student\)
\begin{itemize}
\item Project operators: picks cerntain columns
\end{itemize}
\(\pi_Apply1,Apply2,...,Applyn(Expression)\)
\(\pi_ID apply\)
\(\pi_ID,StudentName(\sigma_GPA > 3.7 Student)\)
\begin{enumerate}
\item cross-product:
\label{sec:orgheadline5}
\begin{itemize}
\item E1 * E2
\end{itemize}
\item natural join
\label{sec:orgheadline6}
\item theta join
\label{sec:orgheadline7}
\end{enumerate}
\subsection{Set operators, renaming, notation}
\label{sec:orgheadline11}
\begin{enumerate}
\item Union operator
\label{sec:orgheadline9}
\begin{itemize}
\item List of college and student names
\end{itemize}
\item Intersection operator
\label{sec:orgheadline10}
\begin{itemize}
\item Join:
\$E1 \join E2
\end{itemize}
\textbf{*}
\end{enumerate}
\section{SQL}
\label{sec:orgheadline16}
\subsection{Basic select statement}
\label{sec:orgheadline13}
\subsection{Table variables and set operators}
\label{sec:orgheadline15}
\begin{enumerate}
\item Union, intersect, except
\label{sec:orgheadline14}
\end{enumerate}
\end{document}
